%-------------------------------------------------------------------------------
%	SECTION TITLE
%-------------------------------------------------------------------------------
\cvsection{Selected Projects}


%-------------------------------------------------------------------------------
%	CONTENT
%-------------------------------------------------------------------------------
\begin{cventries}

%---------------------------------------------------------
  \cventry
    {HTML/CSS, Typescript, and GLSL} % Role
    {Personal Website} % Event
    {London, UK} % Location
    {Jun. 2020 - Present} % Date(s)
    {
      \begin{cvitems} % Description(s)
        \item {Written pages using \href{https://reactjs.org/}{reactJs}, \href{https://nextjs.org/}{nextJs}, \href{https://threejs.org/}{threeJs}, \href{https://sass-lang.com/}{sass},  \href{https://postcss.org/}{postcss}, \href{https://pugjs.org/api/getting-started.html}{pugJs} and \href{https://www.khronos.org/webgl/}{WebGL}.}
        % \item {\href{https://weilueluo.com}{https://weilueluo.com} includes a \href{https://weilueluo.com}{3D landing page}, \href{https://weilueluo.com/about.html}{about page}, and \href{https://weilueluo.com/rss.html}{RSS feeds page}.}
        \item {Deployed an \href{https://developer.mozilla.org/en-US/docs/Web/HTTP/CORS}{CORS} proxy server for RSS feeds using AWS Lambda.}
        \item {Automated integration using \href{https://github.com/features/actions}{Github Actions}.}
        % \item {Project can be found at \href{https://github.com/Redcxx/personal-website}{https://github.com/Redcxx/personal-website}.}
      \end{cvitems}
    }

  %---------------------------------------------------------
  \cventry
    {C\#} % Role
    {First Person Virtual Reality Game - Bouncing Ray} % Event
    {London, UK} % Location
    {Mar. 2022 - Apr. 2022} % Date(s)
    {
      \begin{cvitems} % Description(s)
        \item {Lead a team of 4 to build a first person Virtual Reality game based on the ubiq framework (\href{https://github.com/UCL-VR/ubiq}{\textit{https://github.com/UCL-VR/ubiq}}).}
      \end{cvitems}
    }

%---------------------------------------------------------
  \cventry
    {HTML/CSS, and Javascript} % Job title
    {Collaborate Live} % Organization
    {Manchester, UK} % Location
    {Jan. 2019 - Mar. 2019} % Date(s)
    {
      \begin{cvitems} % Description(s)
        \item {Led a team of 7 students to build a web application for a code-based interview.}
        %  Designed and implemented all core functionalities including real-time chatting, coding, and drawing with collaborators, supports online code compilation and web-embed terminal.
        \item {Awarded 1\textsuperscript{st} place in most appealing idea and highest technical quality prize out of 37 teams.}
      \end{cvitems}
    }
    
%---------------------------------------------------------
  % \cventry
  %   {Sole Developer - Python} % Role
  %   {Machine Learning} % Event
  %   {London, UK} % Location
  %   {Sep. 2020 - Sep. 2022} % Date(s)
  %   {
  %     \begin{cvitems} % Description(s)
  %       \item {\textbf{Time Series Prediction:} Generate Monophonic and Polyphonic music. \textbf{Image Classification:} Transfer learning based on a pre-trained resnet. \textbf{Image Denosing:} Encoder-Decoder network that doubled PSNR compared to traditional techniques. \textbf{Others:} Implemented linear/polynomial/logistic regression; SVM, kSVM; clustering algorithms such as KMeans and Meanshift.}
  %     \end{cvitems}
  %   }
  
  %---------------------------------------------------------
  % \cventry
  %   {Sole Developer - Python} % Role
  %   {Local Nameserver} % Event
  %   {London, UK} % Location
  %   {Feb. 2022 - Mar. 2022} % Date(s)
  %   {
  %     \begin{cvitems} % Description(s)
  %       \item {Written a local nameserver that able to resolve any IPv4 address according to \href{https://datatracker.ietf.org/doc/html/rfc1034}{RFC 1034} and \href{https://datatracker.ietf.org/doc/html/rfc1035}{RFC 1035} specification. The project can be found at \href{https://github.com/Redcxx/local-nameserver}{\textit{https://github.com/Redcxx/local-nameserver}}.}
  %     \end{cvitems}
  %   }

    %---------------------------------------------------------
  \cventry
    {Java} % Role
    {Map Reduce Utilities} % Event
    {Manchester, UK} % Location
    {Feb. 2020 - Mar. 2020} % Date(s)
    {
      \begin{cvitems} % Description(s)
        \item {Developed programs for performing common tasks such as computing tfidf for large document database using Apache Hadoop\'s Map Reduce.}
      \end{cvitems}
    }
  
  %---------------------------------------------------------
  % \cventry
  %   {Sole Developer - Python} % Role
  %   {Poisson Image Editing} % Event
  %   {London, UK} % Location
  %   {Dec. 2021 - Jan. 2022} % Date(s)
  %   {
  %     \begin{cvitems} % Description(s)
  %       \item {Implemented the \href{https://www.cs.jhu.edu/~misha/Fall07/Papers/Perez03.pdf}{Poisson Image Editing paper}. Including techniques such as naive filling; import \& mixing gradient; texture flattening, colour isolation \& illumination. The project can be found at \href{https://github.com/Redcxx/poisson-image-editing}{\textit{https://github.com/Redcxx/poisson-image-editing}}.}
  %     \end{cvitems}
  %   }

  %---------------------------------------------------------
\cventry
  {Java} % Job title
  {MP3 Music Player \& Flight Management System} % Organization
  {Manchester, UK} % Location
  {Mar. 2019 - Jun. 2019} % Date(s)
  {
    \begin{cvitems} % Description(s)
      \item {A MP3 player with GUI and custom database, supports functionalities such as play, pause, restart, and upload custom songs.}
      \item {A flight management system with GUI and custom database, supports functionalities such as booking, reschedule, cancel and swap flights.}
      \item {One of three students who managed to complete this coursework over last 20 years.}
    \end{cvitems}
  }

  %---------------------------------------------------------
  \cventry
    {Java} % Role
    {Propositional Logic Utilities} % Event
    {Manchester, UK} % Location
    {Oct. 2019 - Nov. 2019} % Date(s)
    {
      \begin{cvitems} % Description(s)
        \item {A tool for parsing and manipulating propositional logic formulas, supports conversion to normal forms, push negations, tautology and contradiction checks and truth table generation. The project can be found at \href{https://github.com/Redcxx/PropositionalLogicUtils}{https://github.com/Redcxx/PropositionalLogicUtils}.}
      \end{cvitems}
    }

  %---------------------------------------------------------
  \cventry
    {Python} % Role
    {Image Downloader} % Event
    {Manchester, UK} % Location
    {Jun. 2019 - Aug. 2019} % Date(s)
    {
      \begin{cvitems} % Description(s)
        \item {Designed and implemented an API supports automatic login, multithreading downloading, various format parsing and filter images, along with a graphical user interface. The project can be found at \href{https://github.com/Redcxx/Pikax}{\textit{https://github.com/Redcxx/Pikax}}.}
      \end{cvitems}
    }
  
    
  %---------------------------------------------------------
  % \cventry
  % {Sole Developer - Python} % Job title
  % {Photographic Mosaic} % Organization
  % {Manchester, UK} % Location
  % {Aug. 2019 - Aug. 2019} % Date(s)
  % {
  %   \begin{cvitems} % Description(s)
  %     \item {Built a library for generating photographic mosaic, capable of processing and matching over 30k images within 3 minutes, supports various RGB-based and LAB-based color difference algorithms. The project can be found at \href{https://github.com/Redcxx/Mosaic-Pics}{https://github.com/Redcxx/Mosaic-Pics}.}
  %   \end{cvitems}
  % }




%---------------------------------------------------------
% \cventry
% {} % Role
% {Other} % Event
% {} % Location
% {} % Date(s)
% {
%   \begin{cvitems} % Description(s)
%     \item {More projects can be found at my github \href{https://github.com/Redcxx}{\textit{https://github.com/Redcxx}}.}
%   \end{cvitems}
% }


\end{cventries}
