%-------------------------------------------------------------------------------
% SECTION TITLE
%-------------------------------------------------------------------------------
\cvsection{个人项目}

%-------------------------------------------------------------------------------
% CONTENT
%-------------------------------------------------------------------------------
\begin{cventries}

%---------------------------------------------------------
\cventry
{HTML/CSS、Typescript 和 GLSL} % Role
{个人网站} % Event
{英国伦敦} % Location
{2021年9月 - 现在} % Date(s)
{
\begin{cvitems} % Description(s)
\item {实现的功能:(1) 基于 AWS Lambda 部署的带有 \href{https://developer.mozilla.org/en-US/docs/Web/HTTP/CORS}{CORS} 代理服务器支持的 RSS feed。 (2) 使用 \href{https://github.com/features/actions}{Github Actions} 进行自动化集成。 (3) 使用 ThreeJs 和自定义着色器构建的 3D 落地页。 (4) 现代、响应式、移动友好的 CSS。 (5) 搜索引擎优化。}
\end{cvitems}
}

\cventry
{HTML/CSS 和 Javascript} % Job title
{Collaborate Live} % Organization
{英国曼彻斯特} % Location
{2019年1月 - 2019年3月} % Date(s)
{
  \begin{cvitems} % Description(s)
    \item {带领7名学生为面向代码的面试构建了一个Web应用程序。功能包括实时编码、绘图、编译和终端。}
    % \item {Designed and implemented functionalities such as real-time chatting, coding, and drawing, online code compilation and web-embed terminal.}
    \item {在37支队伍中获得最吸引人的想法和最高技术质量奖(4个奖项中的2个),获得第1名。}
  \end{cvitems}
}
%---------------------------------------------------------
\cventry
{Python - 独立开发} % Role
{泊松图像编辑} % Event
{英国伦敦} % Location
{2021年12月 - 2022年1月} % Date(s)
{
\begin{cvitems} % Description(s)
\item {实现了\href{https://www.cs.jhu.edu/~misha/Fall07/Papers/Perez03.pdf}{泊松图像编辑论文},包括naive filling、import\&mixing gradient、texture flattening、colour isolation \& illumination等技术。该项目位于\href{https://github.com/Redcxx/poisson-image-editing}{\textit{https://github.com/Redcxx/poisson-image-editing}}。}
\end{cvitems}
}

\end{cventries}